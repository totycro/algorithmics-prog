\documentclass[,%fontsize=11pt,%
			paper=a4,% 
			%DIV12, % mehr text pro seite als defaultyyp
			DIV12,
			%DIV=calc,%
			%twoside=false,%
			liststotoc,
			bibtotoc,
			draft=false,% final|draft % draft ist platzsparender (kein code, bilder..)
			%titlepage,
			numbers=noendperiod
			]{scrartcl}


\usepackage[utf8]{inputenc}
\usepackage[T1]{fontenc}
\usepackage[english]{babel}


%\usepackage{vaucanson-g}
%\usepackage{amssymb}
%\usepackage{latexsym}

% for color-highlighted code
%\usepackage{color} % for grey comments
%\usepackage{alltt}

%\usepackage[doublespacing]{setspace}
\usepackage[onehalfspacing]{setspace}
%\usepackage[singlespacing]{setspace}
\usepackage{tabularx}
%\usepackage{hyperref}
\usepackage{comment}
\usepackage{color}
\usepackage{url}      % for urls
\usepackage{multicol}
\usepackage{float}
\usepackage{caption}
\usepackage{amsmath}
\usepackage{mathtools}
\usepackage{amssymb}

\usepackage{soul}

\newcommand{\mst}{\textbf{MST}}
\newcommand{\kmst}{\textit{k}-\mst}

\definecolor{grey2}{gray}{.90}
\sethlcolor{grey2}
\newcommand{\ilc}[1]{\hl{\texttt{#1}}} % ilc = inline code

\title{Report on the implementation of the \kmst\ problem as integer linear program}
\author{Bernhard Mallinger \\ 0707663 \and Christof Schmidt \\ deine matr nr?}

\begin{document}

\maketitle

\section{Problem description}

The \kmst\ problem is a variation of the Minimum Spanning Tree (\mst) problem for undirected graphs $G = (V, E, w)$, including a weight function $w(e) : E \rightarrow \mathbb{R}^+_0$.
The goal of both problems is to find a tree of minimum cost, but instead of requiring it to span the whole graph as in the classical \mst, for the \kmst, the tree has to span a given number of $k$ arbitrary nodes.
This restriction turns the formerly tractable problem into an $NP$-complete one.

\section{General formulation}

This section describes all variables and constraints, that the formulations for connectivity share.

In order to represent the graph, we chose to use a Boolean array \ilc{edges} containing a directed variants of the originally undirected edges.
This will simplify the management of flow later on.
\ilc{edges} contains $2|E|$ elements, where the first half represents the edges in one direction and the other half the opposite ones.
Since trees are acyclic, we can require that only one of the two directed edges that are derived from an undirected edge in the original graph are chosen:
\begin{equation}
 \ilc{edges}[i] + \ilc{edges}[i + |E|] \leq 1\quad\forall~0\leq i \leq |E|
\end{equation}

Furthermore, to avoid circles in the set of solutions, we require that each vertex has at most one incoming edge (\eqref{eq:oneIncoming}) with a special treatment of the artificial root node, which must not have any incoming edge at all (\eqref{eq:root0}) but exactly one outgoing node (\eqref{eq:root1}). 
As the connectivity constraints take effect later, this will ensure that once a node is reached from the root node, there cannot be a path back it (which would form a circle), since then the number of incoming edges would exceed 1.

\begin{equation}\label{eq:oneIncoming}
	\sum_{\mathclap{i\,\in\,\ilc{incomingEdges(j)}}} \ilc{edges}[i] = 1 \quad \forall~0\leq j \leq |V|
\end{equation}
\begin{equation}\label{eq:root0}
	\sum_{\mathclap{i\,\in\,\ilc{incomingEdges(0)}}} \ilc{edges}[i] = 0
\end{equation}
\begin{equation}\label{eq:root1}
	\sum_{\mathclap{i\,\in\,\ilc{outgoingEdges(0)}}} \ilc{edges}[i] = 1 
\end{equation}

Finally, it is still necessary to state the size of the tree to be calculated.
The problem statement requires the tree to have exactly $k$ nodes, which for trees means that there are $k-1$ edges.
Since this formulation also includes an artificial root node, that is not part of the actual problem, we need to allow a further edge, resulting in $k-1+1=k$ edges in total.
\begin{equation}
	\sum_{i=0}^{2|E|} \ilc{edges}[i] = k
\end{equation}

\section{Connectivity formulations}

\subsection{Single Commodity Flow Formulation}
\subsection{Multi Commodity Flow Formulation}
\subsection{Sequential Formulation}

\section{Results}
obj value, running times, number of b\&b nodes

\section{Interpretation of results}

\end{document}
