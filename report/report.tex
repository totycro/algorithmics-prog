\documentclass[,%fontsize=11pt,%
			paper=a4,% 
			%DIV12, % mehr text pro seite als defaultyyp
			DIV14,
			%DIV=calc,%
			%twoside=false,%
			liststotoc,
			bibtotoc,
			draft=false,% final|draft % draft ist platzsparender (kein code, bilder..)
			%titlepage,
			numbers=noendperiod
			]{scrartcl}


\usepackage[utf8]{inputenc}
\usepackage[T1]{fontenc}
\usepackage[english]{babel}


%\usepackage{vaucanson-g}
%\usepackage{amssymb}
%\usepackage{latexsym}

% for color-highlighted code
%\usepackage{color} % for grey comments
%\usepackage{alltt}

%\usepackage[doublespacing]{setspace}
\usepackage[onehalfspacing]{setspace}
%\usepackage[singlespacing]{setspace}
\usepackage{tabularx}
%\usepackage{hyperref}
\usepackage{comment}
\usepackage{color}
\usepackage{url}      % for urls
\usepackage{multicol}
\usepackage{float}
\usepackage{caption}
\usepackage{amsmath}
\usepackage{mathtools}
\usepackage{amssymb}

\usepackage{soul}

\newcommand{\mst}{\textbf{MST}}
\newcommand{\kmst}{\textit{k}-\mst}

\definecolor{grey2}{gray}{.90}
\sethlcolor{grey2}
\newcommand{\ilc}[1]{\hl{\texttt{#1}}} % ilc = inline code


\newcommand{\commodity}{j}
\newcommand{\vertex}{j}
\newcommand{\edge}{i}
%\newcommand{\forallEdges}{\ensuremath{0 \leq \edge \leq 2|E|}}
%\newcommand{\forallEdges}{\ensuremath{\forall \edge \in \{0,\:\dotsc,\:2|E|-1\}}}
\newcommand{\forallEdges}{\ensuremath{\forall \edge \in \{1, \dotsc, 2|E|\}}}
\newcommand{\forallFirstEdges}{\ensuremath{\forall \edge \in \{1, \dotsc, |E|\}}}

\title{Report on the implementation of the \kmst\ problem as integer linear program}
\author{Bernhard Mallinger \\ 0707663 \and Christof Schmidt \\ 0627958}

\begin{document}

\maketitle

\section{Problem description}

The \kmst\ problem is a variation of the Minimum Spanning Tree (\mst) problem for undirected graphs $G = (V, E, w)$, including a weight function $w(e) : E \rightarrow \mathbb{R}^+_0$.
The goal of both problems is to find a tree of minimum cost, but instead of requiring it to span the whole graph as in the classical \mst, for the \kmst, the tree has to span a given number of $k$ arbitrary nodes.
This restriction turns the formerly tractable problem into an \textit{NP}-complete one.

\section{General formulation}

This section describes all variables and constraints, that the formulations for connectivity share. It deals mostly with representing the graph as well as ensuring acyclicity.

In order to represent the graph, we chose to use a Boolean array \ilc{edges} containing a directed variants of the originally undirected edges.
This will simplify the management of flow later on.
\ilc{edges} contains $2|E|$ elements, where the first half represents the edges in one direction and the other half the opposite ones.
Since trees are acyclic, we can require that only one of the two directed edges that are derived from an undirected edge in the original graph are chosen:
\begin{equation}
 \ilc{edges}[i] + \ilc{edges}[i + |E|] \leq 1\quad\forallFirstEdges
\end{equation}

Furthermore, to avoid circles in the set of solutions, we require that each vertex has at most one incoming edge (\eqref{eq:oneIncoming}) with a special treatment of the artificial root node, which must not have any incoming edge at all (\eqref{eq:root0}) but exactly one outgoing node (\eqref{eq:root1}). 
As the connectivity constraints take effect later, this will ensure that once a node is reached from the root node, there cannot be a path back it (which would form a circle), since then the number of incoming edges would exceed 1.
In order to add the constraints only for the relevant and existing edges, we implemented two auxiliary functions: \ilc{incomingEdges}, which returns for a node \ilc{j} the set of indices in the \ilc{edges}-array which correspond to edges incoming to \ilc{j}, and \ilc{outgoingEdges}, which works analogously for outgoing edges.

\begin{equation}\label{eq:oneIncoming}
	\sum_{\mathclap{i\,\in\,\ilc{incomingEdges(j)}}} \ilc{edges}[i] = 1 \quad \forall j \in V
\end{equation}
\begin{equation}\label{eq:root0}
	\sum_{\mathclap{i\,\in\,\ilc{incomingEdges(0)}}} \ilc{edges}[i] = 0
\end{equation}
\begin{equation}\label{eq:root1}
	\sum_{\mathclap{i\,\in\,\ilc{outgoingEdges(0)}}} \ilc{edges}[i] = 1 
\end{equation}

Finally, it is still necessary to state the size of the tree to be calculated.
The problem statement requires the tree to have exactly $k$ nodes, which for trees means that there are $k-1$ edges.
Since this formulation also includes an artificial root node, that is not part of the actual problem, we need to allow a further edge, resulting in $k-1+1=k$ edges in total.
\begin{equation}
	\sum_{i=1}^{2|E|} \ilc{edges}[i] = k
\end{equation}

\section{Connectivity formulations}

The central task of this exercise is to ensure the connectivity of the tree, which can be achieved with any of the following formulations.
The sections are split up into the essential core formulations and strengthening constraints, which are not necessary for correctness but lead to a tighter formulation.

\subsubsection{Strengthening Constraints}
All models were subject to the following constraint, that implies a tighter model. It forces all nodes that do not have an incoming edges to not have any outgoing nodes. Unfortunately, that does not guarantee connectedness, since it still allows circles.
\begin{equation}\label{eq:StrAll}
	\sum_{\mathclap{i \in \ilc{incomingEdges(j)}}} \ilc{edges}[i] * k
 \quad \geq \quad
	\sum_{\mathclap{i \in \ilc{outgoingEdges(j)}}} \ilc{edges}[i]
\quad \forall \ilc{j} \in V \setminus \{0\}
\end{equation}

\subsection{Single Commodity Flow Formulation}

The idea of this formulation is to produce flow at the root node, of which every vertex of the solution consumes one unit.
This consumption however can only take place if the flow is delivered to the node on a path that consists of solution edges.
If such a path from the root node to every solution node exists, it is ensured that these vertices are connected.

The concrete modelling is based upon an array of flow values for each edge named \ilc{flow\_scf}.
The variables of this array are constrained to take nonnegative values up to $k$, where the respective value in \ilc{edges} serves as activation flag: A flow greater than 0 is only allowed in case the edge is taken:
\begin{equation}\label{eq:scf1}
	0 \leq \ilc{flow\_scf}[\edge] \leq k * \ilc{edges}[\edge] \quad \forallEdges 
\end{equation}

The artificial root node $0$ sends out $k$ units of flow:\nolinebreak
\begin{equation}
	\sum_{\mathclap{i\,\in\,\ilc{outgoingEdges(0)}}} \ilc{flow\_scf}[i] = k
\end{equation}

Flow must be conserved, every vertex of the solution sends out as much as it receives minus the one it consumes. In valid solutions, this constraint is only fulfilled by vertices, that are part of the solution. Those differ from the others by having exactly one incoming edge, which is leveraged as activation flag in this formulation:

\begin{equation}
	\sum_{\mathclap{i\,\in\,\ilc{incomingEdges(j)}}} \ilc{flow\_scf}[i] \quad - \quad
	\sum_{\mathclap{i\,\in\,\ilc{outgoingEdges(j)}}} \ilc{flow\_scf}[i] \quad = \quad 
	\sum_{\mathclap{i\,\in\,\ilc{incomingEdges(j)}}} \ilc{edges\_scf}[i] 
	\quad \forall j\,\in\,V
\end{equation}


\subsubsection{Strengthening constraints}

In \eqref{eq:scf1}, the maximum value for the flow is limited by $k$. However since it's a priori deducible that the first edge can only belong to the set of edges leaving the artificial root node, we can tighten the maximum to $k-1$ for all other nodes, as the source of this edge can at most send out this many units of flow.
\begin{equation}\tag{\ref{eq:scf1}$^\prime$}
	0 \leq \ilc{flow\_scf}[i] \leq k * \ilc{edges}[i] \quad \forall i \in \ilc{outgoingEdges(0)} 
\end{equation}
\begin{equation}\tag{\ref{eq:scf1}$^{\prime\prime}$}
	0 \leq \ilc{flow\_scf}[i] \leq (k-1) * \ilc{edges}[i] \quad \forallEdges, i \notin \ilc{outgoingEdges(0)}
\end{equation}

As further optimisation, we tried to restrict the flow variables to integer values, since in valid solutions, they have to be integer. This turned out to have negative effects on the runtime.


\begin{comment}
	- flow on each edge
	- 0 <= flow\_i <= k/k-1 * edge
	- root note emits
	- every node eats/flow conservation
\end{comment}


\subsection{Multi Commodity Flow Formulation}
The Multi Commodity Flow Formulation is based on the Single Commodity Flow Formulation, but instead of single flow, it defines a flow for each vertex separately, called a commodity. Formally the $flow$ is now $\ilc{flow[k][i]} \in \left\{0,1\right\}$ , which is the flow for commodity k, that is destined for vertex k, on the edge $i \in \ilc{edges}$; The flow is sent out from the root, forwarded by each node that isn't the destination, and finally consumed by the destination node.

\subsubsection{Additional (re-used) variables}
The \ilc{incomingEdgesSum[\commodity]} is defined as the sum of all edges ending in vertex \ilc{\commodity} that are part of the \kmst.
\begin{equation}\label{eq:mfcInEdges}
	\ilc{incomingEdgesSum[\commodity]} = \sum_{\mathclap{i \in \ilc{incomingEdges(\commodity)}}} \ilc{edges}[i]
\quad \forall \ilc{\commodity} \in V \setminus \{0\}
\end{equation}


\subsubsection{Constraints}

This constraint implies the sending of the commodity for each vertex that is part of the \kmst\ (=has one incoming edge).
\begin{equation}\label{eq:mfcSend}
	\sum_{\mathclap{\edge \in \ilc{outgoingEdges(0)}}} \ilc{flow[\commodity][\edge]} = \ilc{incomingEdgesSum[\commodity]}
\quad \forall \ilc{\commodity} \in V  \setminus \{0\}
\end{equation}

Each vertex receives his commodity (if part of \kmst):
\begin{equation}\label{eq:mfcRec}
	\sum_{\mathclap{\edge \in \ilc{incomingEdges[j]}}} \ilc{flow[\commodity][\edge]} 
 = \ilc{incomingEdgesSum[\commodity]}
\quad \forall \ilc{\commodity} \in V  \setminus \{0\}
\end{equation}

Each vertex forwards all other commodities:
\begin{equation}\label{eq:mfcForward}
\begin{split}
\sum_{\mathclap i \in \ilc{incomingEdges(v)}} \ilc{flow[\commodity][i]} = \sum_{ i \in \ilc{outgoingEdges(v)}} \ilc{flow[\commodity][i]} 
\quad \forall \ilc{\commodity}, \ilc{v} \in V  \setminus \{0\}
, \ilc{\commodity} \neq \ilc{v}
\end{split}
\end{equation}

Flow is restricted to selected edges:

\begin{equation}\label{eq:mfcFlowEdge}
 \ilc{flow[\commodity][\edge]} \leq \ilc{edges[\edge]}
 \quad \forallEdges 
\quad \forall \ilc{\commodity} \in V  \setminus \{0\}
\end{equation}

\subsubsection{Strengthening Constraints}

There cannot be flow to the root node:
\begin{equation}\label{eq:mfcNoFlow}
 \ilc{flow[\commodity][i]} = 0
\quad \forall i \in \ilc{incomingEdges(0)}  
\quad \forall \ilc{\commodity} \in V
\end{equation}


\subsection{Sequential Formulation}
The sequential formulation, proposed by Miller, Tucker and Zemlin (thus, referred to as MTZ) is  based on an additional variable \ilc{u[\vertex]} $\in \{0..u_{max}\}$ for each node \vertex. It gives the tree an order, in which \ilc{u[0]} has the value 0, each node of the \kmst\ has a an increasing value along the path from the root. Nodes, that are not in the \kmst\ get the value $u_{max}$, which is k in our case.

\subsubsection{Constraints}
The main constraint is only active when \ilc{edges[\edge]} is selected, and thus 1. Otherwise the $u_{max}$ makes sure the constraint always holds. If the edge is selected, it forces the \ilc{u}-value of the start of the edge to be at least one less than the end of the edge, thus inducing an order on the nodes. Along a path from the root to a leaf, the u values will increase strictly monotonic.
%\begin{equation}
	\begin{gather}
	\ilc{u[v1]} + \ilc{edges[\edge]} <= \ilc{u[v2]} + ( ( 1 - \ilc{edges[\edge]}) * u_{max} ) \nonumber \\
	\forall \ilc{v1}| \edge \in \ilc{outgoingEdges(v1)}
	\quad
	\forall \ilc{v2} | \edge \in \ilc{incomingEdges(v2)}
	\quad \forallEdges
\label{eq:mztOrder}
\end{gather}
%\ \ \forall e \in \ilc{edges}
%\end{equation}

This constraint forces all u values for unused nodes (which have no incoming nodes) to the value $u_{max}$, so that by above rule \eqref{eq:mztOrder} they cannot have outgoing edges. This constraint is only needed if strengthening rule \eqref{eq:StrAll} is not present.
\begin{equation}\label{eq:mztUnused}
 u_{max}*\left(1 - \sum_{ i \in \ilc{incomingEdges(v)}} \ilc{edges[i]} \right) \leq \ilc{u[\vertex]}
 \quad \forall \vertex \in V  \setminus \{0\}
\end{equation}

\subsubsection{Strengthening Constraints}

For all non artificial nodes, we know that the u value has to be at least 1. (Not possible in combination with \eqref{eq:mztMaxU})
\begin{equation}\label{eq:mztNodes1}
 \ilc{u[\vertex]} \geq 1
 \quad \forall \vertex \in V  \setminus \{0\}
\end{equation}

If strengthening rule \eqref{eq:StrAll} is present, this constrain replaces \eqref{eq:mztUnused}. It reduces the choices for the solver by forcing the u value for unused nodes to 0 and thus makes rule \eqref{eq:mztMaxU} possible.
\begin{equation}\label{eq:mztUnused2}
 \left( u_{max}* \sum_{ i \in \ilc{incomingEdges(v)}} \ilc{edges[i]} \right) \geq \ilc{u[\vertex]}
 \quad \forall \vertex \in V  \setminus \{0\}
\end{equation}

For small values of  $u_{max}$, this constraint tightens the selection of the \ilc{u} values, since for all valid solutions, there is an assignment of $u$ such that the sum of all $u$ values are at most $\sum_1^k{k}$.
\begin{equation}\label{eq:mztMaxU}
	\sum_{\vertex \in V}{\ilc{u[\vertex]}} \leq \frac{k * (k+1)}{2}
\end{equation}

The value of $u_{max}$ was also subject to a strengthening or relaxing. The original proposal has $u_{max}$ set to the number of nodes, which makes sense for a TSP that is a path. Therefore we tried to use $n$, but in our case, we only have $k$ nodes, so $k$ is a more natural and tighter candidate. Furthermore, we tested larger numbers such as $n*k$. Our idea was, that the actual value of \ilc{u} does not matter, but only the ordering is important (a node "closer" to the root has to have a lower \ilc{u} value). So a larger $u_{max}$ value gives the solver more freedom to find a suitable choice.

\section{Results}
We executed all the tests on the provided ADS Server on a single core, without status output and in the production compile mode. All but the very large instances have been run several times to equalize random deviations of execution times. The last two columns are the results of the formulation without the strengthening constraints.


\subsection{Single Commodity Flow Formulation}

\begin{tabular}{||c||cccr|cr||}

\hline
test     & k & objective  & branch-n-bound & running   & BnB nodes & running\\
instance &   & function   & nodes          & time (sec)& (w/o SC)  & time w/o SC \\
\hline

g01.dat	&	2	&	46	&	0	&	0.012	&	0	&	0.012	\\
g01.dat	&	5	&	477	&	0	&	0.016	&	0	&	0.018	\\
g02.dat	&	4	&	373	&	5	&	0.019	&	5	&	0.019	\\
g02.dat	&	10	&	1390	&	80	&	0.056	&	80	&	0.098	\\
g03.dat	&	10	&	725	&	36	&	0.145	&	36	&	0.098	\\
g03.dat	&	25	&	3074	&	62	&	0.236	&	62	&	0.234	\\
g04.dat	&	14	&	909	&	216	&	0.483	&	216	&	0.353	\\
g04.dat	&	35	&	3292	&	69	&	0.478	&	69	&	0.459	\\
g05.dat	&	20	&	1235	&	119	&	0.447	&	119	&	0.415	\\
g05.dat	&	50	&	4898	&	286	&	1.125	&	286	&	0.934	\\
g06.dat	&	40	&	2068	&	1799	&	11.069	&	1799	&	11.498	\\
g06.dat	&	100	&	6705	&	1587	&	11.215	&	1587	&	11.510	\\
g07.dat	&	60	&	1335	&	282	&	19.275	&	282	&	18.390	\\
g07.dat	&	150	&	4534	&	977	&	48.275	&	977	&	49.095	\\
g08.dat	&	80	&	1620	&	461	&	73.010	&	461	&	70.590	\\
g08.dat	&	200	&	5787	&	461	&	161.705	&	461	&	144.505	\\

\hline

\end{tabular}

\subsection{Multi Commodity Flow Formulation}

\begin{tabular}{||c||cccr|cr||}
\hline
test     & k & objective  & branch-n-bound & running   & BnB nodes & running\\
instance &   & function   & nodes          & time (sec)& (w/o SC)  & time w/o SC \\
\hline

g01.dat	&	2	&	46	&	0	&	0.073	&	0	&	0.074	\\
g01.dat	&	5	&	477	&	0	&	0.054	&	0	&	0.073	\\
g02.dat	&	4	&	373	&	9	&	0.224	&	9	&	0.306	\\
g02.dat	&	10	&	1390	&	0	&	0.212	&	0	&	0.227	\\
g03.dat	&	10	&	725	&	0	&	1.848	&	0	&	1.985	\\
g03.dat	&	25	&	3074	&	0	&	2.383	&	0	&	2.411	\\
g04.dat	&	14	&	909	&	67	&	11.294	&	67	&	10.264	\\
g04.dat	&	35	&	3292	&	0	&	12.378	&	0	&	12.610	\\
g05.dat	&	20	&	1235	&	3	&	37.684	&	3	&	37.666	\\
g05.dat	&	50	&	4898	&	0	&	31.528	&	0	&	31.894	\\
g06.dat	&	40	&	2068	&	231	&	2,180.080	&	231	&	2,134.190	\\
g06.dat	&	100	&	6705	&	9	&	2,016.450	&	9	&	1,996.000	\\
g07.dat	&	60	&	-	&	-	&	-	&	-	&	-	\\
g07.dat	&	150	&	-	&	-	&	-	&	-	&	-	\\
g08.dat	&	80	&	-	&	-	&	-	&	-	&	-	\\
g08.dat	&	200	&	-	&	-	&	-	&	-	&	-	\\

\hline
\end{tabular}


\subsection{Sequential Formulation}

\begin{minipage}{\linewidth} % to keep note at bottom on same page
\begin{tabular}{||c||cccr|cr||}
\hline
test     & k & objective  & branch-n-bound & running   & BnB nodes & running\\
instance &   & function   & nodes          & time (sec)& (w/o SC)  & time w/o SC \\
\hline
g01.dat	&	2	&	46	&	0	&	0.004	&	0	&	0.008	\\
g01.dat	&	5	&	477	&	20	&	0.016	&	15	&	0.022	\\
g02.dat	&	4	&	373	&	43$^{(**)}$	&	0.026$^{(**)}$	&	91$^{(*)}$	&	0.036$^{(*)}$	\\
g02.dat	&	10	&	1390	&	217	$^{(**)}$&	0.085	$^{(**)}$&	665	&	0.131	\\
g03.dat	&	10	&	725	&	9	&	0.116	&	601	&	0.589	\\
g03.dat	&	25	&	3074	&	1062	&	0.699	&	1494	&	2.221	\\
g04.dat	&	14	&	909	&	86	&	0.440	&	510	&	1.862	\\
g04.dat	&	35	&	3292	&	595	&	1.656	&	10537$^{(*)}$	&	8.928$^{(*)}$	\\
g05.dat	&	20	&	1235	&	132	&	0.596	&	2362	&	4.668	\\
g05.dat	&	50	&	4898	&	3016	&	4.068	&	12769	&	30.080	\\
g06.dat	&	40	&	2068	&	8023	&	15.780	&	12555	&	129.180	\\
g06.dat	&	100	&	6705	&	28492	$^{(**)}$&	90.020$^{(**)}$	&	23966$^{(**)}$	&	100.190$^{(**)}$	\\
g07.dat	&	60	&	1335	&	5098$^{(*)}$	&	40.000$^{(*)}$	&	68448$^{(*)}$	&	863.860	$^{(*)}$\\
g07.dat	&	150	&	4534	&	2210$^{(**)}$	&	42.680$^{(**)}$	&	-	&	-	\\
g08.dat	&	80	&	1620	&	1050	&	58.270	&	-	&	-	\\
g08.dat	&	200	&	5787	&	26345$^{(**)}$	&	260.360$^{(**)}$	&	94744$^{(*)}$	&	1,620.380$^{(*)}$	\\
\hline
\end{tabular}

$^{(*)}$ = Result for $u_{max} = n*k$ instead of the usual $k$.


$^{(**)}$ = Result for $u_{max} = n$ instead of the usual $k$.
\end{minipage}

\begin{figure}[htb]
	\centering	\includegraphics[width=1.00\textwidth]{chart.png}
	\caption{Comparison of runtimes between the models}
	\label{fig:chart}
\end{figure}


\section{Interpretation of results}
Chart \ref{fig:chart} shows big differences in the performances of the different formulations. These difference are present for all instance sizes and grow with increasing complexity. The SCF model has the best runtime, closely followed and sometimes beaten by the improved MTZ model. So these two formulation seem suitable to handle this kind of problem. Our original MTZ model and the MCF model however did not perform well and their runtime grew too fast for the larger instances.


The MCF formulation wasn't even computable for instances 7 and 8 due to the quadratic number of constraints. This however is not just a deficiency but is caused by the tightness of the formulation. The low number of branch-and-bound nodes compared to the other formulations shows that optimal solutions can often be obtained with very little branching, at the expense of long runtimes for solving the relaxation.


The biggest impact on the performance was the introduction of \eqref{eq:StrAll}. It was initially thought as a solely strengthening constraint for all three models, but in the case of MTZ, where \eqref{eq:mztUnused} ensures the same property (no outgoing edges without incoming edges), replacing the constraint resulted in a significany performance boost. It had no such effect on SCF and MCF, we think that this is already implied equally close by the flow formulations.


The other strengthening constraint usually did not yield significant performance changes, which seems reasonable since they only tighten the model for a few constraints. 


We were surprised that small changes on the model led to very different running time. Usually the impact was worse for some and better for other instances. This effect is particularly visible for the sequential formulation. It was very sensitive to the value of $u_{max}$, which can be anything above k, and had very different results for different choices, but no value was dominant for all (or even most) instances. You can see this in the table for MZT, where we took the best results from different $u_{max}$ values. It may be that this problem is caused by the choice of branching - where a small change could lead to a different branching in the beginning and thus a very different branch tree. Still, there seems to be a correlation between the input size and the best choice of $u_{max}$. From this we conclude that the proper formulation for a problem strongly depends on the input data and may be developed for each situation individually (or configured dynamically).


In addition, even for the same model and instance, the runtime fluctuated unusually strong, even for the large instances. We compensated this effect by executing several rounds of tests.

\end{document}
